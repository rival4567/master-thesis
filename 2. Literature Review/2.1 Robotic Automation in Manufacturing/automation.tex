Industrial robots were first used for repetitive tasks and material handling.
The first industrial robot, \textbf{Unimate} was deployed by General Motors in 1961. It weighed two tons and worked on assembly lines, autonomously lifting heaving objects and welding car parts. 
Since then robots have evolved to become versatile device that can perform complex tasks, learn from experience, communicate through devices, and collaborate with human workers. \cite{firstrobot}

Industrial robots are the right solution for high-volume production process for their efficiency, uptime and quality. \cite{jrautomation} As the manufacturing industry moves smaller batch production, cobots would be much more useful and flexible in a smart workcell. Cobots are vastly more advanced and affordable than industrial robots. \cite{jrautomation2}
Common use cases of robotic automation in manufacturing include material handling \cite{gambao2012new,SKIBNIEWSKI1992251}, welding \cite{tarn2011robotic}, assembly \cite{ji2021learning}, pick-and-place \cite{shah2021design}, palletizing \cite{lee2021intelligent}, and even metal sheet bending \cite{Uhrhan1995}.

Automation is particularly useful in industrial where there is a risk for human operators.
In \cite{10381692}, robotic automation has been implemented for the manufacturing of footwears which is a hazardous environment for human workers.
In it a robotic workcell consisting of three robots is controlled and coordinated through \hyperref[acro:ROS]{ROS}.

With the \hyperref[acro:CPS]{CPS}, the machines in the robotic workcell communicate and share data with each other with which production process is improved. Also, whole production process
can be monitored remotely.\cite[page 105]{li2020robotics} With the \hyperref[acro:AM]{AM} technologies, design to production time is vastly decreased. \cite[page 116]{li2020robotics}. The benefits of automation can often outweigh the initial costs. This way Industry 4.0 is transforming manufacturing sector. 

% \cite{kassowrobotsblog} Robotic automation has revolutionized the manufacturing industry by enhancing productivity, precision, and safety. Industrial robots are capable of performing a variety of tasks such as welding, assembly, material handling, and more recently, metal sheet bending. The Kassow robot, known for its high flexibility and precision, is an example of a modern industrial robot that can be effectively integrated into automated workcells.

% Recent studies have demonstrated the benefits of robotic automation in reducing cycle times, minimizing errors, and improving product quality. For instance, Kang et al. (2017) showed how integrating robots into manufacturing lines can enhance efficiency and consistency. However, the successful implementation of robotic automation requires careful consideration of factors such as system design, control algorithms, and sensor integration.
