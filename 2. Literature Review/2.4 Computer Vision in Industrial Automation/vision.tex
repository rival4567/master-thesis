
Computer vision (\hyperref[acro:CV]{CV}) techniques have played an important role in promoting the information, digitization, and intelligence of industrial manufacturing systems.
\hyperref[acro:CV]{CV} applications include inspection, identification and process control with precision, reliability, and scalability.
In recent years, advancements in camera technology, image processing algorithms, and \hyperref[acro:CV]{CV} techniques have significantly increased the capabilities of vision systems in manufacturing systems.
The most common methods of \hyperref[acro:CV]{CV} are feature detection, recognition, segmentation, and \hyperref[acro:3D]{3D} modeling. \cite{9761203}

CV is necessary for the successful automation of the bending process. \hyperref[acro:KR]{KR} should be able to accurately detect and handle metal sheets and perform precise bending operations. With \hyperref[acro:CV]{CV}, images can be analyzed in real-time, allowing for immediate feedback and adjustment in the automated process.
This capability is crucial for maintaining high production speeds while ensuring precision.
There are many examples of \hyperref[acro:CV]{CV} being used in industry environments, like in the sorting and classification of food products \cite{BARNES2010339, THROOP2005281,BURGOSARTIZZU2010138}, monitoring and safety management of construction projects \cite{PANERU2021103940} or the automated traffic monitoring system. \cite{7892717,COIFMAN1998271}

The smart features can be added to regular cameras for industrial environments, enhances their functionality for various automation tasks. \cite{BREZANI2022298} However, it requires image processing algorithms to run on separate hardware. 
The \hyperref[acro:VISOR]{VISOR}\textsuperscript{\textregistered} vision sensor does not need a \hyperref[acro:PC]{PC} or \hyperref[acro:PLC]{PLC} to run. A \hyperref[acro:PC]{PC} or laptop is required only in order to configure the
\hyperref[acro:VISOR]{VISOR}\textsuperscript{\textregistered} vision sensor. \cite[page 23]{visor_user_manual}