% Integrating robotics, computer vision, and a user-friendly interface into a cohesive system is critical for the successful automation of the bending process. This integration ensures that the robot can accurately detect and handle metal sheets, perform precise bending operations, and provide real-time feedback to the user.

% Studies by Patel et al. (2017) and Garcia et al. (2021) have demonstrated the benefits of such integrated systems in various manufacturing applications. The integration process involves developing robust communication protocols, ensuring synchronization between components, and optimizing the system for real-time operation. The use of open-source platforms and modular design principles can facilitate this integration and enhance system flexibility.

Computer vision (CV) techniques have played an important role in promoting the information, digitization, and intelligence of industrial manufacturing systems.
CV applications include inspection, identification and process control with precision, reliability, and scalability.
In recent years, advancements in camera technology, image processing algorithms, and \hyperref[tab:acronyms]{CV} techniques have significantly increased the capabilities of vision systems in manufacturing systems.
The most common methods of \hyperref[tab:acronyms]{CV} are feature detection, recognition, segmentation, and \hyperref[tab:acronyms]{3D} modeling. \cite{9761203}


1.One of the critical aspects of implementing computer vision in industrial settings is the ability to process and analyze images in real-time, allowing for immediate feedback and adjustment in automated processes. This capability is crucial for maintaining high production speeds while ensuring precision 2.

2D Vision Systems in Industrial Applications

2D vision systems, like the Sensopart VISION camera, are widely used in industrial automation due to their ability to capture high-resolution images and detect features on flat surfaces. These systems are particularly effective in applications such as sheet detection, where they identify edges, holes, and other critical features required for further processing 3.

The Sensopart VISION camera, known for its high accuracy and robust performance, is an example of how 2D vision systems can be applied in sheet detection. This camera is capable of detecting even subtle variations in sheet features, enabling precise alignment and processing in automated bending operations. Such systems are invaluable in ensuring that the correct sheets are used in production and that they are correctly positioned, reducing errors and improving overall process efficiency 4.

Advancements in 2D Vision and Image Processing

Recent advancements in image processing algorithms have further enhanced the functionality of 2D vision systems in industrial settings. Techniques such as edge detection, pattern recognition, and feature extraction have improved the ability to identify and classify objects with high accuracy. These developments are crucial for tasks such as detecting sheet features in complex industrial environments, where lighting conditions and surface variations can pose significant challenges 5.

Moreover, the integration of machine learning with computer vision has opened new possibilities for adaptive and intelligent systems that can learn from data and improve performance over time. This approach is particularly useful in applications where variability in the manufacturing process is high, as it allows the vision system to adapt to changes and maintain high levels of accuracy 6.

Conclusion

In conclusion, computer vision, particularly 2D vision systems like the Sensopart VISION camera, plays a critical role in industrial automation. These systems enable precise detection and processing of features on sheets, contributing to the efficiency and accuracy of automated bending processes. As technology continues to advance, the integration of more sophisticated image processing and machine learning techniques will further enhance the capabilities of vision systems in industrial applications.

This review provides a concise overview of the role of computer vision in industrial automation, with a specific focus on 2D vision systems and their applications in sheet detection, such as with the Sensopart VISION camera.