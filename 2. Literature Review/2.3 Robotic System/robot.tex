The real benefit of robots is taking over the three Ds, the dull, the dirty and dangerous jobs. \cite{jordan2016robots}
The functionality of a robotic system during any step of control should include three principal performance
features in the cognitive process: perception, recognition and decision-making. 
It is obvious that the autonomy of the whole robotic system directly corresponds to sensory equipment, 
processing sensory information and decision algorithms. \cite{HAVLIK2011327}

Robots systems are used in industrial environments for assembling parts, painting cars or welding operations. \cite{SathishKumar2023, Wakizako}. Robots could be arranged in assembly lines to carry out a particular repetitive task. With the development in robotic perception and algorithms, robotic systems could be placed in a workcell to perform multiple tasks sequentially.
Sensory systems play an important role to develop intelligent robotic systems. \cite{Wakizako}. Using smart sensors, a single robotic arm could perform various tasks by making decisions.

There are various  kind of robotic arms available in the market. These include industrial robots as well as cobots. Cobots are generally more sensitive and easier to program. They are also more safe to work with humans as compared to industrial robots. Industrial robots require an enclosed space for safety reasons. However, they are more durable and the speed is high. \cite{10201199}
Robotic arm also comes in different number of \hyperref[acro:DOF]{DOF}. 6-axis robotic arms are mostly common, but an additional axis from a 7-axis cobot is advantageous in our case for collision free trajectory planning in a large workspace. 

A good example of cobot is the 7-axis cobot by Kassow Robots. This sophisticated robot can mimic human movements and offers an extensive range of motion that allows it to handle intricate tasks, navigate tight spaces, and maintain consistent product quality in an industrial setting.
These cobots not only increase efficiency and productivity but also promote safety.
They can operate in industrial environments and reduces the risks associated with humans as they are collaborative. 
With the ability to work tirelessly nonstop, these cobots ensure continuous production and uniform quality standards.
\cite{kassowrobotsblog}

\subsection{Robot Operating System}
\label{subsec:ROS}
The Robot Operating System (ROS) is a set of software libraries and tools which help to build robot applications. 
Because it is open-source, there is flexibility where and how to use \hyperref[acro:ROS]{ROS}, as well as the freedom to customize if it requires. \cite{rosblog}
\hyperref[acro:ROS]{ROS} is utilized to build, control or simulate various kind of robots, from mobile robots to robotic arms. \cite{koubaa2017robot}
It could also be used in partnership with a simulation environment like Gazebo. This will fasten the development process
and also make decisions before actually buying a robot. 

In \cite{takaya2016simulation}, a simulation environment is created for mobile
robots using \hyperref[acro:ROS]{ROS} and gazebo and later tested with the real robot, showcasing the usability of the \hyperref[acro:ROS]{ROS} for the development process.
In \cite{qian2014manipulation}, a manipulator is simulated for the pick and place operation.
There are numerous robots build using \hyperref[acro:ROS]{ROS} and is used extensively by both hobbyists and robotic developers and is a powerful tool.