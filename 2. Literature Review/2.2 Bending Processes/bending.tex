Sheet metal bending is a process in which bends are formed using a combination of a punch and a die. This process is used to create large number of mechanical products such as furniture panels, shelves, cabinets, housing for electro-mechanical devices etc. \cite{alvaautomated}
The project partner for this thesis \textit{i.e.} \textbf{mech-tron GmbH \& Co. KG} excels in the manufacturing of housing systems for electronic and embedded equipment. In this thesis, manufacturing of one of these sheet metal housing systems will be automated by means of robotic workcell.



The automation of bending processes has been a focus of research due to its potential to improve efficiency and reduce labor costs. Traditional bending methods involve manual operations that are time-consuming and subject to human error. Automated bending machines, equipped with CNC (Computer Numerical Control) systems, have alleviated some of these issues but still require significant human intervention for tasks such as loading and unloading.

Research by Lee et al. (2018) and Smith et al. (2021) has explored the development of fully automated bending systems using industrial robots. These studies demonstrate the feasibility of using robots to handle metal sheets and perform bending operations with high precision. However, challenges remain in integrating these systems with other manufacturing processes and ensuring their reliability in a production environment.