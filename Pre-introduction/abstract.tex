\begin{abstract}

    This thesis demonstrates the use of a robotic workcell for
    the automation of a bending machine. The main objectives
    are to automate the bending process of a AMADA bending machine and
    enhance accuracy of bent sheets.
    The robotic workcell consists of a
    7-axis collaborative robot, two VISOR\textsuperscript{\textregistered}
    vision sensors, and a web-based UI. The robotic arm is used for the automatic
    loading and unloading of metal sheets in the bending machine.
    To achieve this, a VISOR\textsuperscript{\textregistered} camera is mounted
    on the robot for sheet detection and robotic perception. The second VISOR\textsuperscript{\textregistered}
    is used for the angle measurement of bent part and quality control.
    A web-based UI monitors the robot movement in the workcell which
    acts as a digital twin. 
    The benefits include reduction in manual labor and increased
    throughput, with improved precision and efficiency.
    This work provides valuable insights for the rapid development
    and deployment in the industry automation.
    
\vspace{1em}
\noindent \textbf{Keywords:} Robotic Workcell, Automation, 
Metal Sheet Bending, 7-axis robotic arm, VISOR\textsuperscript{\textregistered}
Computer Vision, ROS, Web UI,
Digital Twin
    
\end{abstract}

