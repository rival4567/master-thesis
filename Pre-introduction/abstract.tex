\begin{abstract}

    This thesis demonstrates the use of a robotic workcell for
    the automation of a bending machine. The main objective
    is to automate the bending process using a single handling robot. The robot is performing all bending steps that would normally be taken by a human operator on the bending machine\textemdash while also keeping high accuracy and consistency in production.
    The robotic workcell consists of a
    7-axis collaborative robot, two \hyperref[acro:VISOR]{VISOR}\textsuperscript{\textregistered}
    vision sensors, and a web-based UI. The robotic arm is used for the automatic
    loading, and unloading of metal sheets in the bending machine.
    To achieve this, a \hyperref[acro:VISOR]{VISOR}\textsuperscript{\textregistered} camera is mounted
    on the robot for sheet detection and perception of the workcell. The second \hyperref[acro:VISOR]{VISOR}\textsuperscript{\textregistered} camera
    is used for checking the bent part and quality control.
    A web-based \hyperref[acro:UI]{UI} monitors the robot trajectories in the workcell and allows for programming the robot for a new sheet metal variant.
    The benefits include reduction in manual labor and increased
    throughput, with better precision than a human operator. The robotic workcell allows flexibility in production by quickly adapting to a new part.
    This work provides valuable insights for the rapid development
    and deployment of a robotic workcell in the industrial automation for low-volume, customizable and flexibile production.
    
\vspace{1em}
\noindent \textbf{Keywords:} Robotic Workcell, Automation, 
Metal Sheet Bending, 7-axis robotic arm, \hyperref[acro:VISOR]{VISOR}\textsuperscript{\textregistered}
Computer Vision, \hyperref[acro:ROS]{ROS}, Web \hyperref[acro:UI]{UI}
    
\end{abstract}

