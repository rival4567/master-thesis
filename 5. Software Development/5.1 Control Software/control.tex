
\begin{figure}
    \centering
    \begin{tikzpicture}[node distance=1.4cm,
        stage/.style={draw, rectangle, minimum width=2.5cm, minimum height=1cm, align=center},
        ]
        % Nodes
        \node (unloading) [stage] {\hyperref[par:unloading]{Unloading}};
        \node (alignment) [stage, below of=unloading] {\hyperref[par:alignment]{Alignment}};
        \node (bending) [stage, below of=alignment] {\hyperref[par:bending]{Bending}};
        \node (checking) [stage, below of=bending] {\hyperref[par:checking]{Checking}};
        \node (loading) [stage, below of=checking] {\hyperref[par:loading]{Loading}};
        % Arrows
        \draw [arrow] (unloading) -- (alignment);
        \draw [arrow] (alignment) -- (bending);
        \draw [arrow] (bending) -- (checking);
        \draw [arrow] (checking) -- (loading);
        \draw [arrow] (checking) .. controls ++(-2.25,0) and ++(-2.25,0) .. (alignment);
    \end{tikzpicture}    
    \caption{Five stages of bending process}
    \label{fig:stages-bp}
\end{figure}

After analyzing the current manual bending process on the bending machine by an operator, five main stages are established for the completion of a single bending operation of a sheet metal part. To make the bending process automated, these five stages of bending operation needs to be replaced by the robotic workcell. Figure \ref{fig:stages-bp} shows these fives stages: unloading, alignment, bending, checking and loading.

\begin{enumerate}
    \item \par{\textbf{Unloading}}
    \label{par:unloading}
    In the first stage, a sheet metal part is to be separated from a stack of metal sheets in unloading station. The advantage of
    arranging the raw sheets in stacks is that many sheet metal parts can be stored in a little space. The sheet thickness is
    also checked in advance before loading the sheets in unloading station by operator and parts outside the tolerance are sorted out. A single sheet is to be picked up by a gantry robot using a magnetic gripper and placed in the unloading station gripper. Then the unloading station gripper turns 180\textdegree around using a pneumatic swivel unit. The sheet metal part is now ready to be picked up by the robot.
    
    
    \item \par{\textbf{Alignment}}
    \label{par:alignment}
    In the next stage, the sheet metal part is to be placed in the bending machine. The robot scans the sheet metal part grasped by the unloading station gripper in the unloading station. Once the features of the part is detected using the robotic camera, trajectories are planned to pick up the sheet from the unloading station gripper. After picking up the sheet, path is planned to the correct bending station for the bending operation. The bending station is chosen based on the sheet metal part variant and the ease of first bending operation. The bending machine sets the backgauges according to the bending machine program for a specific sheet metal variant. The sheet metal part is then aligned in the 5-axis backgauges of the bending machine. The sheet is now ready to be bent. 
    
    \item \par{\textbf{Bending}}
    \label{par:bending}
    The bending machine is operated by a saved program for a specific sheet metal type variant. The bending machine is lowered until the bending of the part starts. During the bending process, there should be two different design variants depending on the sheet
    variant. With small sheets, the robot lets go of the sheet and the bending punch keeps holding the sheet in place. After the bending is complete, the robot regrip the bent sheet. This is easier to achieve with the robot programming. In order to prevent twisting in the case of larger parts, it is essential to hold the parts throughout the entire bending process. The challenge here lies in programming a robot movement
    whose path and path speed matches the movement of the sheet metal, which in turn depends on the
    lowering speed of the bending punch.
    
    \item \par{\textbf{Checking}}
    \label{par:checking}
    A camera system is used to record the bending angle of the sheet metal part after each individual
    bending process. The angle is used to determine the quality of the bending operation. After the inspection is complete, the process continues  for a good part and the sheet is taken to the bending machine for the next bending process. If the bending angle does not lie within the tolerances, the part is thrown away and the bending machine is resetted for a new sheet metal part using the terminal operating robot. The loop restart again for a new sheet. If many rejects occur in a very short time, the entire automated bending process is to be stopped.
    If all the bending on a sheet metal type variants is finished, it is taken to the storage station for loading.
    
    \item \par{\textbf{Loading}}
    \label{par:loading}
    The fully bent sheet metal parts are to be placed in the storage station which is a shelf of empty drawers. A regrasping of the finished bent part is also done using the unloading station gripper, if it allows for an easier placement of the part in the drawer. After the loading of a part is complete, the bending process is finished for a single sheet metal variant. The cycle continues with a new sheet metal part.
    
\end{enumerate}

\vspace{1\baselineskip}
A flowchart for the robot unit is developed based on the above-mentioned stages for the automated bending process of the sheet metal part. This
flowchart, which is still partly abstract, enables the individual activities of the \hyperref[acro:KR]{KR1410} to be identified and
the interactions between various subsystems to be described. The description of the process and the definition of the overall requirements in section \ref{sec:requirements} represent an interacting process that inevitably had to be iterated through. This means that the results
of the bending process were revised several times. The final result of these iterations regarding the
activities to be performed by the \hyperref[acro:KR]{KR1410} robot can be seen in flowchart \ref{tab:flowchart}. 

\begin{longtable}{c}
    \footnotesize
    \begin{tikzpicture}[node distance=3.5cm]
        \node (start) [startstop] {Start};
        \node (proc0) [process, below of=start, yshift=1cm] {Initialisation of camera, \hyperref[acro:PLC]{PLC} and configuration of storage system parameters};
        \node (loop)  [process, below of=proc0, yshift=1.5cm] {Loop};
        \node (nodeint) [nodepage, left of=loop] {3};
        \node (dec0) [decision, below of=loop] {Is calibration \\ requested from operator?};
        \node (intersection0) [coordinate, below of=dec0] {};
        \node (proc1) [process, right of=dec0, xshift=3cm] {Perform hand-eye calibration};
        \node (dec1) [decision, below of=intersection0, yshift=0.5cm] {Is drawer of shelf open?};
        \node (proc2) [process, right of=dec1, xshift=3cm] {Open empty drawer of shelf};
        \node (intersection1) [nodepage, below of=dec1] {1};

        \draw [arrow] (start) -- (proc0);
        \draw [arrow] (proc0) -- (loop);
        \draw [arrow] (loop) -- (dec0);
        \draw [arrow] (dec0) -- node[anchor=east, yshift=0.3cm, xshift=0.35cm] {Yes} (proc1);
        \draw [black, thick] (dec0) -- node[anchor=south, yshift=-0.3cm, xshift=-0.35cm] {No} (intersection0);
        \draw [black, thick] (proc1) |- (intersection0);
        \draw [arrow] (intersection0) -- (dec1);
        \draw [arrow] (dec1) -- node[anchor=east, yshift=0.3cm, xshift=0.35cm] {No} (proc2);
        \draw [arrow] (dec1) -- node[anchor=south, yshift=-0.3cm, xshift=-0.45cm] {Yes} (intersection1);
        \draw [arrow] (proc2) |- (intersection1);
        \draw [arrow] (nodeint) -- (loop);
    \end{tikzpicture} \\

    \begin{tikzpicture}[node distance=3.5cm]
        \footnotesize

        \node (intersection1) [nodepage] {1};
        \node (proc3) [processorange, below of=intersection1, yshift=1.75cm] {Wait until sheet metal part is available at unloading station};
        \node (proc4) [process, below of=proc3, yshift=1.5cm] {Detect sheet metal part using camera};
        \node (point7) [coordinate, left of=proc4, xshift=-0.5cm] {};
        \node (proc5) [process, below of=proc4, yshift=1.5cm] {Pick the sheet metal part from the unloading station gripper};
        \node (proc6) [processorange, below of=proc5, yshift=1.5cm] {Wait until bending machine is ready};
        \node (point5) [coordinate, left of=proc6] {};
        \node (proc7) [process, below of=proc6, yshift=2cm] {Perform bending operation};
        \node (proc8) [process, below of=proc7, yshift=2cm] {Inspect bending angle using second camera};
        \node (dec2) [decision, below of=proc8, yshift=0cm] {Is bending angle within tolerance?};
        \node (proc9) [process, right of=dec2, xshift=3cm] {Discard sheet metal part};
        \node (intersection2) [nodepage, below of=proc9, yshift=2cm] {3};
        \node (point1) [coordinate, below of=dec2, yshift=0.5cm] {};
        \node (point2) [coordinate, right of=point1, xshift=3cm] {};
        \node (dec3) [decision, below of=point2, yshift=1cm] {Is part finished?};
        \node (intersection3) [nodepage, below of=dec3, yshift=0.5cm] {2};
        \node (point3) [coordinate, left of=dec3, xshift=0cm] {};
        \node (point4) [coordinate, below of=point3, yshift=3cm] {};
        \node (dec4) [decision, left of=point4, xshift=0.5cm] {Is regrasping of part required?};
        \node (point6) [coordinate, left of=dec4] {};
        \node (point8) [process, below of=dec4, yshift=0cm] {Place sheet metal part in unloading station gripper};
        \node (point9) [coordinate, left of=point8, xshift=-0.5cm] {};


        \draw [arrow] (intersection1) -- (proc3);
        \draw [arrow] (proc3) -- (proc4);
        \draw [arrow] (proc4) -- (proc5);
        \draw [arrow] (proc5) -- (proc6);
        \draw [arrow] (proc6) -- (proc7);
        \draw [arrow] (point5) -- (proc6.west);
        \draw [arrow] (proc7) -- (proc8);
        \draw [arrow] (proc8) -- (dec2);
        \draw [arrow] (dec2) -- node[anchor=east, yshift=0.3cm, xshift=0.35cm] {No} (proc9);
        \draw [arrow] (proc9) -- (intersection2);
        \draw [black, thick] (dec2) -- (point1);
        \draw [black, thick] (point1) -- node[anchor=east, yshift=0.3cm] {Yes} (point2);
        \draw [arrow] (point2) -- (dec3);
        \draw [black, thick] (dec3) -- node[anchor=west, yshift=0.3cm, xshift=-0.5cm] {No} (point3);
        \draw [black, thick] (point3) -- (point4);
        \draw [arrow] (point4) --  (dec4);
        \draw [black, thick] (point5) -- (point6);
        \draw [black, thick] (point6) -- node[anchor=west, yshift=0.3cm, xshift=-0.5cm] {No} (dec4);
        \draw [black, thick] (dec4) -- node[anchor=east, yshift=0.2cm, xshift=0cm] {Yes} (point8);
        \draw [black, thick] (point8) -- (point9);
        \draw [black, thick] (point9) -- (point7);
        \draw [arrow] (point7) --  (proc4.west);

        % \draw [arrow] (dec2) -- node[anchor=south, yshift=-0.1cm, xshift=-0.45cm] {Yes} (dec3);
        \draw [arrow] (dec3) -- node[anchor=south, yshift=0cm, xshift=-0.45cm] {Yes} (intersection3);


    \end{tikzpicture} \\

    \begin{tikzpicture}[node distance=2cm]
        \footnotesize
        \node (intersection3) [nodepage] {2};
        \node (proc10) [process, below of=intersection3] {Place the sheet in the drawer of shelf};
        \node (dec5) [decision, below of=proc10, yshift=-1.5cm] {Is current drawer of shelf filled?};
        \node (proc11) [process, right of=dec5, xshift=4.5cm] {Close the drawer of shelf};
        \node (dec6) [decision, below of=proc11, yshift=-2cm] {Is storage station\\ full?};
        \node (intersection4) [nodepage, below of=dec5, yshift=-2cm] {3};
        \node (stop) [startstop, below of=dec6, yshift=-1.75cm] {Stop};


        \draw [arrow] (intersection3) -- (proc10);
        \draw [arrow] (proc10) -- (dec5);
        \draw [arrow] (proc11) -- (dec6);
        \draw [arrow] (dec5) -- node[anchor=east, yshift=0.3cm, xshift=0cm] {Yes} (proc11);
        \draw [arrow] (dec5) -- node[anchor=east, yshift=0.3cm, xshift=0cm] {No} (intersection4);
        \draw [arrow] (dec6) -- node[anchor=west, yshift=0.3cm, xshift=0cm] {No} (intersection4);
        \draw [arrow] (dec6) -- node[anchor=south, yshift=-0.2cm, xshift=-0.4cm] {Yes} (stop);


    \end{tikzpicture} \\
    \\
    \caption{Control program flowchart of robot unit} \label{tab:flowchart}
\end{longtable}

The flowchart \ref{tab:flowchart} is used to create the control software for the \hyperref[acro:KR]{KR1410} robot in teach pendant. A number of variables are also required to coordinate the bending process with the \hyperref[acro:PLC]{PLC} as seen in the flowchart. The required data is shared between \hyperref[acro:PLC]{PLC} and \hyperref[acro:KR]{KR1410} as shown in table \ref{tab:kr1410-to-plc} and \ref{tab:plc-to-kr1410}.

The unloading station gripper is also used for the regrasping of the bent part. This allows the robot to grip the sheet metal at a new position, if necessary, for the subsequent bending as otherwise collisions between the robot and the bending machine may occur. To achieve this, robot brings back the bent metal part to the unloading station gripper at a different configuration, scans the bent sheet, detect the sheet pattern and at last picks up the sheet from a new position.




