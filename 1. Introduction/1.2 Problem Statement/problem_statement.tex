Manual bending of metal sheets is labor-intensive and can 
lead to human errors, resulting in inconsistent quality
and inefficiency. 
The automation of this process with a collaborative robot poses
several challenges, which includes the precise detection and processing of sheet metal parts, the accurate execution of
the bend operation and the coordination of
the robotic system with other systems in the workcell.

A bending process has five stages, namely, unloading, alignment, bending, checking and loading.
Despite advancements in robotics and automation, many automated bending cells still require manual intervention or are only partially automated. Or they use seperate robotic system to operate on one of the stage of the bending process, hence requiring multiple robotic mechanisms. The challenge with this research lies in automating all the five stages of the bending process with a single handling robot. Robotic perception is required in this case for advanced error handling and collision free path planning in the workspace.
% This robotic workcell will have automatic loading and unloading
% of metal sheets in the \hyperref[acro:AMADA]{AMADA} bending machine and will use
% computer vision to detect
% and measure the bend angle of metal sheets.
% In addition, a user-friendly interface will be developed for
% monitoring the robotic workcell.