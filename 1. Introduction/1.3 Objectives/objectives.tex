The objectives of this research are as follows:
\begin{enumerate}
    \item Designing and developing a robotic workcell capable of bending metal sheets autonomously. The workcell includes \hyperref[acro:AMADA]{AMADA} HFP80-25 NT as the bending machine to be automated, \hyperref[acro:KR]{KR1410} as the robot, \hyperref[acro:SensoPart]{SensoPart} vision sensors as cameras, an unloading station to get sheets from the operator and a shelf for placing the sheets.
    \item Integration of computer vision technology to detect metal sheets accurately for pick-up and measure bending angle. One \hyperref[acro:VISOR]{VISOR\textsuperscript{\textregistered}} vision sensor mounted on the \hyperref[acro:KR]{KR1410} will be used for sheet detection and second \hyperref[acro:VISOR]{VISOR\textsuperscript{\textregistered}} placed in robot workspace for measuring bending angle. Communication between the vision sensor and \hyperref[acro:KR]{KR} needs to be setup using Telegram.
    \item Building a web interface for the visualization and monitoring of robotic workcell in real-time. The web application interfaces with the \hyperref[acro:KR]{KR} through \hyperref[acro:ROS]{ROS} Bridge library. The interface uses web development technologies like \hyperref[acro:JS]{ReactJS, NodeJS}, Javascript, \hyperref[acro:HTML5]{HTML} and \hyperref[acro:CSS]{CSS}.
    \item Evaluating the efficiency and accuracy of the robotic workcell in industrial environments and compare it with already existing solution.
\end{enumerate}