\begin{enumerate}
    \item A key requirement for the mobile robot unit is to be able to cope with the limited space available in the
    project partner's production hall. Specifically, this means that only an area of approximately $2 m \times 4.5 m$ is
    available in front of the sheet metal bending machine.

    \item Another point is the operating time of the overall system. The unit should be able to autonomously
    manage a day shift of eight hours and a night shift of six hours without any major personnel
    intervention.

    \item Robot should also be able to handle various sheet metal part variants, i.e. different component sizes and
    geometries. In addition to the flexible handling of different sheet metal parts, the focus is also on the
    simple integration of new variants into the existing system. Some of the sheet metal parts in production are
    of small size. These sheets will have little gripping surface after few bendings. Special considerations needs to be given to the
    gripper design so that robot can handle sheet metal parts of small and medium sizes. (from $60 mm \times 110 mm$ to upto $115 mm \times 220 mm$)
  
    \item 
    The mobile robot unit is to be used as a supplement to manual production. It is therefore necessary
    that it can be set up and dismantled quickly and easily in front of the bending machine.

    \item The hand-eye calibration using a vision sensor could degrade over time due to environmental factors such as temperature changes or vibrations.
    This will decrease the accuracy of robot \hyperref[acro:TCP]{TCP} in positioning and grasping leading to incorrect bending.
    The robot should be able to automatically recalibrate without any long delays when there is a request by the operator.
    
    \item The angle measurement on the bent sheet should be carried out using a camera system
    after each bending process. Based on the measured angle, bending machine should be re-adjusted for the next iteration. The
    necessary values should also be entered at the operator terminal by the terminal operating robot. In addition, a
    continuous real-time status query of the bending machine is required via the operating terminal. For
    the implementation of this subtask, the company is working with \textbf{VisCheck GmbH}, which specializes in
    reading screens using a camera and making entries via a robot.
    
\end{enumerate}






