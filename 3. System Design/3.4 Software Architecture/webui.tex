
\subsubsection{Front-end technologies}
\label{subsubsec:frontend}
Front-end development is the development of visual and interactive elements of a website that users interact with directly. It's a combination
of \hyperref[acro:HTML5]{HTML}, \hyperref[acro:CSS]{CSS} and JavaScript, where \hyperref[acro:HTML5]{HTML} provides the structure, \hyperref[acro:CSS]{CSS} the styling and layout, and JavaScript the dynamic behavior and interactivity.
\cite{frontend}

\paragraph{Reactjs}
\label{par:reactjs}
is the most popular front-end JavaScript library for building user interfaces. React can also render on the server using Node and power mobile
apps using React Native. It uses interfaces out of individual pieces called components written in JavaScript.
\cite{reactjs}

\subsubsection{Backend technologies}
\label{subsubsec:backend}

Backend development refers to the server-side aspect of web development, focusing on creating and managing the server logic, databases,
and \hyperref[acro:API]{APIs}. It involves handling user authentication, authorization, and processing user requests, typically using backend development
languages such as Python, Java, JavaScript (Node.js), and .NET. \cite{backend}

\paragraph{Node.js}
\label{par:nodejs}
is an open-source, cross-platform JavaScript runtime environment that lets developers create servers, web application etc.
The web application for robotic workcell is built and hosted on a Node.js server. \cite{nodejs}

\paragraph{npm}
\label{par:npm}
is a package manager for Node.js. It stands for Node Package Manager. Through this, ROS web tools for server-side runtime are installed like
roslib, ros3djs. \cite{npm}

\subsubsection{ROS Web Tools}
\label{subsubsec:rosweb}

Robot Web Tools are open-source libraries and tools for building web-based robot apps with \hyperref[acro:ROS]{ROS}. Rosbridge, roslibjs, ros3Djs, and visualization-rwt packages
are used for building the web application which are part of ROS Web Tools. \cite{webtools} Rosbridge and visualization-rwt are installed on the \hyperref[acro:ROS]{ROS} workspace
for robotic workcell. Roslibjs and ros3Djs are libraries for building the web application and are installed on the server built on Node.js.

\paragraph{Rosbridge}
\label{par:rosbridge}
The WebSocket makes it possible to open a two-way interactive communication session between the user's browser and a server.
With this API, messages can be sent to a server and received through event-driven responses without having to poll the server for a reply. \cite{websocket}
Rosbridge provides a websocket interface to \hyperref[acro:ROS]{ROS} systems. It will provide interface to front-end technologies like ReactJs which builds the UI
and publishes a web application. Rosbridge suite is a meta-package containing rosbridge, various front end packages for Rosbridge 
like a WebSocket package, and helper packages. \cite{rosbridge}

\paragraph{roslibjs}
\label{par:roslibjs}
provides base dependencies and support libraries for ROS. roslib contains many of the common data structures and tools that are shared across
ROS client library implementations. \cite{roslib} It will provide support to interact with basic \hyperref[acro:ROS]{ROS} functionalities like topics, services, parameter servers and others.

\paragraph{ros3djs}
\label{par:ros3djs}
is the standard JavaScript \hyperref[acro:3D]{3D} visualization manager for ROS. It is build on top of roslibjs and utilizes the power of three.js \cite{threejs}.
Many standard \hyperref[acro:ROS]{ROS} features like interactive markers, \hyperref[acro:URDF]{URDF}, and maps are included as part of this library. \cite{ros3djs}

\paragraph{visualization-rwt}
\label{par:visualization-rwt}
package is a suite of nodes for web based robot visualization. It provides nodes for controlling the robot using MoveIt through web application.
\cite{visualization-rwt}
