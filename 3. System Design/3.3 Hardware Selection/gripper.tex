The gripper was selected according to payload, gripping force and opening width. The
opening width is the dominant selection criterion here, as the gripper must be able to grip both the thin
sheets with thicknesses of around 1 to 3 mm and the handle on the drawers with a width of 15 mm. An
electric gripper was also considered, as the power supply could have been provided via the cable
already integrated in the robot. Due to the higher costs, the double height and the weight, a pneumatic
gripper was chosen.

\subsubsection{Robotic Gripper}
\label{subsubsec:robotic-gripper}
It is possible to know if the gripper is open or closed.

\subsubsection{Unloading Station Gripper}
\label{subsubsec:unloading-gripper}
The pneumatic parallel gripper mounted on the Pneumatic swivel unit.