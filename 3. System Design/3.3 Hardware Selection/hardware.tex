\subsection{Kassow Robot KR1410}



\subsubsection{Increased Safety}

Industrial Robotic Automation significantly enhances workplace safety. Robots, especially cobots like Kassow Robot's 7-axis version, are equipped with advanced safety features and mechanisms. 
They can detect and avoid collisions with people or objects, reducing the risks associated with manual labour in potentially hazardous environments. This commitment to safety exemplifies the primary objective of robotic automation: to protect workers and streamline processes.

\subsubsection{More Flexibility}
Cobots, with their adaptable designs, represent the epitome of flexibility in industrial automation. 
Unlike traditional fixed robots, the 7-axis from Kassow Robots can be quickly reprogrammed and adapted for different tasks. This agility makes them an invaluable asset for industries undergoing frequent changes or those with diverse product lines.

\subsubsection{Cost-effective}
Introducing Industrial Robotic Automation is a wise financial decision for businesses. Cobots, which offer a competitive price point compared to traditional industrial robots, provide significant returns on investment by improving efficiency, reducing waste, and minimizing production downtime. 
The long-term savings, in terms of both time and money, underscore the economic benefits of integrating robotic automation into the production process.

\subsubsection{Great Scalability}
Industrial robots, especially cobots, are inherently scalable. Whether it's adjusting to a new product line or ramping up production volumes, Kassow Robot's 7-axis cobot can be swiftly recalibrated to meet evolving demands. 
This ensures that industries can easily scale their operations without undergoing extensive overhauls or incurring excessive costs.

\subsubsection*{Improved Quality}
Quality assurance is a prominent benefit of Industrial Robotic Automation. Robots, with their precision and consistency, dramatically reduce the margin for errors in production. 
Kassow Robot's 7-axis model ensures that each task—from welding to assembly—is executed with remarkable accuracy, leading to products of superior quality and consistency.

\subsubsection*{Higher Productivity}

Robotic automation directly translates to increased productivity. By automating repetitive and time-consuming tasks, cobots free human workers to concentrate on value-added activities. 
In scenarios where cobots collaborate with human workers, the synergy often leads to faster production cycles, efficient workflows, and overall heightened productivity.

\subsection{VISOR\textsuperscript{\textregistered} Vision Sensor}
\label{sec:visor_3}

According to \hyperref[acro:VISOR]{VISOR}\textsuperscript{\textregistered} \cite[page 22]{visor_user_manual} user manual, the \hyperref[acro:VISOR]{VISOR}\textsuperscript{\textregistered} vision sensor is an optical sensor and is used for the non-contact acquisition or identification of objects.
The vision sensor features a number of different evaluation methods (detectors), with the specific methods
depending on the specific model sensor. The product is designed for industrial use only. The \hyperref[acro:VISOR]{VISOR}\textsuperscript{\textregistered} vision sensor is a cost-effective alternative to conventional image processing systems.

\hyperref[acro:VISOR]{VISOR}\textsuperscript{\textregistered} vision sensor is a suitable choice for the identification and classification of metal sheets for the automated bending process.
These sensors enable precise detection and processing of features on sheets, which leads to the efficiency and accuracy of automated bending process.
Special considerations need to be taken in complex industrial environments, where light reflections or changing extraneous light can distort evaluation results. For this reason, an external light source is used to protect against ambient light.